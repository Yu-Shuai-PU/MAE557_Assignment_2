\documentclass{article}

\usepackage{graphicx}%
\usepackage{subcaption}%
\usepackage{multirow}%
\usepackage{amsmath,amssymb,amsfonts}%
\usepackage{amsthm}%
\usepackage{mathrsfs}%
\usepackage[title]{appendix}%
\usepackage{xcolor}%
\usepackage{textcomp}%
\usepackage{manyfoot}%
\usepackage{booktabs}%
\usepackage{algorithm}%
\usepackage{algorithmicx}%
\usepackage{algpseudocode}%
\usepackage{listings}%
\usepackage{bm}%
\usepackage[normalem]{ulem}  % for \sout
\usepackage{float}
\usepackage{placeins}  % For \FloatBarrier
\usepackage[
colorlinks=true,
linkcolor=blue,
citecolor=red,
urlcolor=magenta
]{hyperref}
\usepackage{geometry}
\geometry{
  left=15mm,
  right=15mm,
  top=10mm,
  bottom=10mm
}
\usepackage{onimage}

%%%% 

%% as per the requirement new theorem styles can be included as shown below
\theoremstyle{plain}
\newtheorem{theorem}{Theorem}%
\newtheorem{proposition}[theorem]{Proposition}%

\theoremstyle{remark} %
\newtheorem{example}{Example}%
\newtheorem{remark}{Remark}%

\theoremstyle{definition} % <--- 修改这里 (定义用 definition 样式)
\newtheorem{definition}{Definition}%

\newcommand{\ip}[2]{\left\langle #1, #2 \right\rangle}
\newcommand{\U}{\mathcal U}
\newcommand{\cwrremark}[1]{\textcolor{blue}{[CWR: #1]}}
\newcommand{\ysremark}[1]{\textcolor{red}{[YS: #1]}}
\newcommand{\solidline}{\rule[0.5ex]{1.5em}{0.6pt}}
\newcommand{\dashedline}{\makebox[1.5em][l]{\rule[0.5ex]{0.3em}{0.6pt}\hspace{0.2em}%
    \rule[0.5ex]{0.3em}{0.6pt}\hspace{0.2em}%
    \rule[0.5ex]{0.3em}{0.6pt}}}

\raggedbottom
%% \unnumbered% uncomment this for unnumbered level heads

\begin{document}

\title{MAE 557 Mini-Project Two}

\author{Yu Shuai}

\maketitle

\section{Governing equations for the 2D compressible lid-driven cavity flow}
\label{sec:problem_1}

Generally speaking, the governing equations of a single-component compressible flow consist of:
\begin{itemize}
\item the continuity equation
\item the momentum equation
\item the energy equation
  \item the equation of state
  \end{itemize}

  Here, since the flow is viscous, \textcolor{blue}{the kinetic energy is inherently not conservative. Consequently, the momentum equationis not secondary conservative.}
  To avoid introducing spurious temperature fluctuations, here \textcolor{blue}{we choose the energy equation to be the equation of internal energy.}

  The mathematical form of governing equations are then given as follows:
  \begin{subequations}
    \label{eq:1}
    \begin{align}
      \label{eq:2}
      \frac{\partial}{\partial t}\rho + \frac{\partial}{\partial x_j}(\rho u_j) &= 0,\\
      \label{eq:3}
      \frac{\partial}{\partial t}(\rho u_i) + \frac{\partial}{\partial x_j}(\rho u_ju_i) &= \frac{\partial}{\partial x_j}\Bigg(-p\delta_{ij} + \mu\bigg(\frac{\partial u_i}{\partial x_j} + \frac{\partial u_j}{\partial x_i} - \frac{2}{3}\frac{\partial u_k}{\partial x_k}\delta_{ij}\bigg)\Bigg),\\
      \label{eq:4}
      \frac{\partial}{\partial t}(\rho e) + \frac{\partial}{\partial x_j}(\rho u_je) &= \frac{\partial}{\partial x_j}\bigg(\lambda\frac{\partial T}{\partial x_j}\bigg) - p\frac{\partial u_k}{\partial x_k} + 2\mu\Bigg(\frac{1}{2}\bigg(\frac{\partial u_i}{\partial x_j} + \frac{\partial u_j}{\partial x_i}\bigg) - \frac{1}{3}\frac{\partial u_k}{\partial x_k}\delta_{ij}\Bigg)^2,\\
      \label{eq:5}
      p &= \rho RT\\
      \label{eq:6}
      e &= \frac{1}{\gamma - 1}RT,
      \end{align}
    \end{subequations}
    where $x_1, x_2\in[0, L]$ are horizontal and vertical spatial coordinates respectively, $\rho$ is the density, $u_i (i = 1, 2)$ are velocity components, $p$ is the pressure, and $e$ is the internal energy.
    $\mu = \rho\nu$ is the dynamical viscosity, $\lambda$ is the thermal diffusivity, $R = 287\,\mathrm{J}/(\mathrm{kg}\cdot\mathrm{K})$ is the specific gas constant, $\gamma = c_p/c_v$ is the ratio of specific heats.

    Initially, the flow is stationary, i.e., $u(x_1, x_2, t = 0) = 0$ for all $x_1, x_2$.
    So, we can assume that the density is initially uniform $\rho(x_1,x_2,t = 0) = \rho_0$ for all $x_1, x_2$.
    The initial pressure and temperature are uniformly given $p(x_1, x_2, t = 0) = p_0 = 1 \mathrm{bar} = 10^5\, \mathrm{Pa}$ and $T(x_1, x_2, t = 0) = T_0 = 300\, \mathrm{K}$.
    From these values, we can utilize the equation of state \eqref{eq:5} to obtain $\rho_0 = 100000/86100 \approx 1.161\, (\mathrm{kg}/\mathrm{m}^3)$

    For the boundary conditions, the horizontal velocity on the top wall satisfy $u_1(x_1, x_2 = L, t) = u_w(t) = U_w\sin(\omega t)$, while the velocities on all other walls are $0$ satisfying the no-slip and no-penetration boundary conditions.
    Since the walls are isothermal, we have $T(x_1 = 0, x_2, t) = T(x_1, x_2 = 0, t) = T(x_1 = L, x_2, t) = T(x_1, x_2 = L, t) = 300\,\mathrm{K}$.
    The boundary conditions of density  are given following the Navier-Stokes characteristic boundary conditions (NSCBC, \cite{Poinsot1992NSCBC}).
    \begin{equation}
      \label{eq:7}
      \frac{\partial \rho}{\partial t} + \frac{\partial}{\partial n}(\rho u_n) + \frac{\partial}{\partial \tau}(\rho u_\tau) = 0.
    \end{equation}
    Here, $n$ is the normal direction to the boundary pointing outwards from the computational domain, and $\tau$ is the tangential direction such that $(n, \tau)$ forms a right-handed local coordinate frame.
    At the left wall, the unit vectors of the lab frame and the local frame are related as $e_n = -e_{x_1}$ and $e_\tau = -e_{x_2}$.
    At the bottom wall, we have $e_n = -e_{x_2}$ and $e_{\tau} = e_{x_1}$.
    At the right wall, we have $e_n = e_{x_1}$ and $e_\tau = e_{x_2}$.
    At the top wall, we have $e_n = e_{x_2}$ and $e_\tau = -e_{x_1}$.
    Thus, the boundary conditions of density at each wall can be specified as follows:
    \begin{subequations}
      \label{eq:8}
      \begin{align}
        \label{eq:9}
        & \frac{\partial \rho}{\partial t}(0, x_2, t) + \frac{\partial(\rho(0, x_2, t)(-u_1(0, x_2, t)))}{\partial (-x_1)} = 0\Longrightarrow \frac{\partial \rho(0, x_2, t)}{\partial t} + \rho(0, x_2, t)\frac{\partial u_1(0, x_2, t)}{\partial x_1} = 0\\
        \label{eq:10}
        & \frac{\partial \rho}{\partial t}(x_1, 0, t) + \frac{\partial(\rho(x_1, 0, t)(-u_2(x_1, 0, t)))}{\partial (-x_2)} = 0\Longrightarrow \frac{\partial \rho(x_1, 0, t)}{\partial t} + \rho(x_1, 0, t)\frac{\partial u_2(x_1, 0, t)}{\partial x_2} = 0\\
        \label{eq:11}
        & \frac{\partial \rho}{\partial t}(L, x_2, t) + \frac{\partial(\rho(L, x_2, t)u_1(L, x_2, t))}{\partial x_1} = 0\Longrightarrow \frac{\partial \rho(L, x_2, t)}{\partial t} + \rho(L, x_2, t)\frac{\partial u_1(L, x_2, t)}{\partial x_1} = 0\\
        & \frac{\partial \rho}{\partial t}(x_1, L, t) + \frac{\partial(\rho(x_1, L, t)u_2(x_1, L, t))}{\partial x_2} + \frac{\partial(\rho(x_1, L, t)(-u_1(x_1, L, t)))}{\partial (-x_1)} = 0\notag\\
        \label{eq:12}
        \Longrightarrow & \frac{\partial \rho(x_1, L, t)}{\partial t} + \rho(x_1, L, t)\frac{\partial u_2(x_1, L, t)}{\partial x_2} + u_1(x_1, L, t)\frac{\partial\rho(x_1, L, t)}{\partial x_1} = 0. 
        \end{align}
      \end{subequations}

      Next, we non-dimensionalize our governing equations.
      Let $\rho^* = \rho/\rho_0$, $x_j^* = x_j/L$ and $u_j^* = u_j/U_w$, we have $t^* = t/t_c = U_wt/L$ and $t_c = L/U_w$, where all quantities with the superscript $*$ are dimensionless.
      The continuity equation \eqref{eq:2} then becomes
      \begin{equation}
        \frac{\partial \rho^*}{\partial t^*} + \frac{\partial}{\partial x_j^*}(\rho^*u_j^*) = 0.
      \end{equation}
      Additionally, if we let $p^* = p/p_c$ where $p_c = \rho_0U_w^2$, then the momentum equation \eqref{eq:3} becomes
      \begin{equation}
        \frac{\partial}{\partial t^*}(\rho^* u_i^*) + \frac{\partial}{\partial x_j^*}(\rho^* u_j^*u_i^*) = -\frac{\partial p^*}{\partial x_i^*} + \frac{1}{\mathrm{Re}}\frac{\partial}{\partial x_j^*}\Bigg(\rho^*\bigg(\frac{\partial u_i^*}{\partial x_j^*} + \frac{\partial u_j^*}{\partial x_i^*} - \frac{2}{3}\frac{\partial u_k^*}{\partial x_k^*}\delta_{ij}\bigg)\Bigg),
      \end{equation}
      where $\mathrm{Re} = U_wL/\nu$ is the Reynolds number with $\nu$ the constant kinematic viscosity.
      Additionally, if we let $e^* = e/e_c$ where $e_c = RT_c/(\gamma - 1)$ and $T_c = T_0 = 300 \,\mathrm{K}$, and let the reference speed of sound $a_c = \sqrt{\gamma RT_c}$, then the internal energy equation \eqref{eq:4} becomes:
      \begin{equation}
        \frac{\partial}{\partial t^*}(\rho^* e^*) + \frac{\partial}{\partial x_j^*}(\rho^* u_j^*e^*) = \frac{\gamma}{\mathrm{Pr}\mathrm{Re}}\frac{\partial}{\partial x_j^*}\bigg(\rho^*\frac{\partial T^*}{\partial x_j^*}\bigg) - \gamma(\gamma - 1)\mathrm{Ma}^2p^*\frac{\partial u_k^*}{\partial x_k^*} + \frac{2\mathrm{Ma}^2\gamma(\gamma - 1)}{\mathrm{Re}}\rho^*\Bigg(\frac{1}{2}\bigg(\frac{\partial u_i^*}{\partial x_j^*} + \frac{\partial u_j^*}{\partial x_i^*}\bigg) - \frac{1}{3}\frac{\partial u_k^*}{\partial x_k^*}\delta_{ij}\Bigg)^2,
      \end{equation}
      where $\mathrm{Pr} = \rho c_p\nu/\lambda$ is the constant Prandtl number with $c_p$ the constant-pressure specific heat and $\lambda$ the thermal conductivity, $\gamma$ the ratio of specific heats, $\mathrm{Ma} = U_w/a_c$ the Mach number with $a_c = \sqrt{\gamma RT_c}$.

      Apart from non-dimensionalizing the equations, we also need to non-dimensionalize the boundary conditions and initial conditions.
      For the boundary conditions, we know that the horizontal velocity on the top wall is given by $u_1(x_1, L, t) = U_w\sin(\omega t)$, so we have $u_1^*(x_1^*, 1, t^*) = \sin((\omega L/U_w) t^*) = \sin(2t^*/Re)$.
      Similarly, the boundary conditions for the dimensionless temperature is $T^*(0, x_2^*, t^*) = T^*(x_1^*, 0, t^*) = T^*(1, x_2^*, t^*) = T^*(x_1^*, 1, t^*) = 1$.
      For the initial conditions, we have $p^*(x_1^*, x_2^*, 0) = p_0/(\rho_0U_w^2) = 1/(\gamma\mathrm{Ma}^2)$.

      Finally, we combine the dimensionless governing equations, initial conditions, physical boundary conditions and characteristic boundary conditions to form the following problem:
      \begin{subequations}
        \label{eq:13}
        \begin{align}
          \label{eq:14}
          & \frac{\partial \rho^*}{\partial t^*} + \frac{\partial}{\partial x_j^*}(\rho^*u_j^*) = 0,\\
          \label{eq:15}
          & \frac{\partial}{\partial t^*}(\rho^* u_i^*) + \frac{\partial}{\partial x_j^*}(\rho^* u_j^*u_i^*) = -\frac{\partial p^*}{\partial x_i^*} + \frac{1}{\mathrm{Re}}\frac{\partial}{\partial x_j^*}\Bigg(\rho^*\bigg(\frac{\partial u_i^*}{\partial x_j^*} + \frac{\partial u_j^*}{\partial x_i^*} - \frac{2}{3}\frac{\partial u_k^*}{\partial x_k^*}\delta_{ij}\bigg)\Bigg),\\
          \label{eq:16}
          & \frac{\partial}{\partial t^*}(\rho^* e^*) + \frac{\partial}{\partial x_j^*}(\rho^* u_j^*e^*) = \frac{\gamma}{\mathrm{Pr}\mathrm{Re}}\frac{\partial}{\partial x_j^*}\bigg(\rho^*\frac{\partial T^*}{\partial x_j^*}\bigg) - \gamma(\gamma - 1)\mathrm{Ma}^2p^*\frac{\partial u_k^*}{\partial x_k^*} + \frac{2\mathrm{Ma}^2\gamma(\gamma - 1)}{\mathrm{Re}}\rho^*\Bigg(\frac{1}{2}\bigg(\frac{\partial u_i^*}{\partial x_j^*} + \frac{\partial u_j^*}{\partial x_i^*}\bigg) - \frac{1}{3}\frac{\partial u_k^*}{\partial x_k^*}\delta_{ij}\Bigg)^2,\\
          \label{eq:17}
          & p^* = \frac{1}{\gamma\mathrm{Ma}^2}\rho^*T^*,\\
          \label{eq:18}
          & e^* = T^*,\\
          \label{eq:19}
          & u_1^*(x_1^*, 1, t^*) = \sin(2t^*/\mathrm{Re}), u_1^*(0, x_2^*, t^*) = u_1^*(x_1^*, 0, t^*) = u_1^*(1, x_2^*, t^*) = 0,\\
          \label{eq:20}
          & u_2^*(x_1^*, 1, t^*) = u_2^*(0, x_2^*, t^*) = u_2^*(x_1^*, 0, t^*) = u_2^*(1, x_2^*, t^*) = 0,\\
          \label{eq:21}
          & T^*(0, x_2^*, t^*) = T^*(x_1^*, 0, t^*) = T^*(1, x_2^*, t^*) = T^*(x_1^*, 1, t^*) = 1,\\
          \label{eq:22}
          & \frac{\partial \rho^*(0, x_2^*, t^*)}{\partial t^*} + \rho^*(0, x_2^*, t^*)\frac{\partial u_1^*(0, x_2^*, t^*)}{\partial x_1^*} = 0,\\
          \label{eq:23}
          & \frac{\partial \rho^*(x_1^*, 0, t^*)}{\partial t^*} + \rho^*(x_1^*, 0, t^*)\frac{\partial u_2^*(x_1^*, 0, t^*)}{\partial x_2^*} = 0,\\
          \label{eq:24}
          & \frac{\partial \rho^*(1, x_2^*, t^*)}{\partial t^*} + \rho^*(1, x_2^*, t^*)\frac{\partial u_1^*(1, x_2^*, t^*)}{\partial x_1^*} = 0,\\
          \label{eq:25}
          & \frac{\partial \rho^*(x_1^*, 1, t^*)}{\partial t^*} + \rho^*(x_1^*, 1, t^*)\frac{\partial u_2^*(x_1^*, 1, t^*)}{\partial x_2^*} + u_1^*(x_1^*, 1, t^*)\frac{\partial\rho^*(x_1^*, 1, t^*)}{\partial x_1^*} = 0,\\
          \label{eq:26}
          & \rho^*(x_1^*, x_2^*, 0) = 1,\\
          \label{eq:27}
          & T^*(x_1^*, x_2^*, 0) = 1,\\
          \label{eq:28}
          & u_1^*(x_1^*, x_2^*, 0) = u_2^*(x_1^*, x_2^*, 0) = 0,\\
          \label{eq:29}
          & \mathrm{Re} = 100, \quad \gamma = 1.4, \quad \mathrm{Ma} = 0.025, \quad \mathrm{Pr} = 0.7.
          \end{align}
        \end{subequations}

        Or, one can rewrite the equations in a componentwise form:
        \begin{subequations}
        \label{eq:30}
        \begin{align}
          \label{eq:31}
          & \frac{\partial \rho^*}{\partial t^*} + \frac{\partial}{\partial x_1^*}(\rho^*u_1^*) + \frac{\partial}{\partial x_2^*}(\rho^*u_2^*)= 0,\\
          \label{eq:32}
          & \frac{\partial}{\partial t^*}(\rho^* u_1^*) + \frac{\partial}{\partial x_1^*}(\rho^* u_1^*u_1^*) + \frac{\partial}{\partial x_2^*}(\rho^* u_2^*u_1^*) = -\frac{\partial p^*}{\partial x_1^*} + \frac{1}{\mathrm{Re}}\frac{\partial}{\partial x_1^*}\Bigg(\rho^*\bigg(\frac{\partial u_1^*}{\partial x_1^*} + \frac{\partial u_1^*}{\partial x_1^*} - \frac{2}{3}(\frac{\partial u_1^*}{\partial x_1^*} + \frac{\partial u_2^*}{\partial x_2^*})\bigg)\Bigg)\notag\\
          & + \frac{1}{\mathrm{Re}}\frac{\partial}{\partial x_2^*}\Bigg(\rho^*\bigg(\frac{\partial u_1^*}{\partial x_2^*} + \frac{\partial u_2^*}{\partial x_1^*}\bigg)\Bigg),\\
          \label{eq:33}
          & \frac{\partial}{\partial t^*}(\rho^* u_2^*) + \frac{\partial}{\partial x_1^*}(\rho^* u_1^*u_2^*) + \frac{\partial}{\partial x_2^*}(\rho^* u_2^*u_2^*) = -\frac{\partial p^*}{\partial x_2^*} + \frac{1}{\mathrm{Re}}\frac{\partial}{\partial x_1^*}\Bigg(\rho^*\bigg(\frac{\partial u_2^*}{\partial x_1^*} + \frac{\partial u_1^*}{\partial x_2^*}\bigg)\Bigg)\notag\\
          & + \frac{1}{\mathrm{Re}}\frac{\partial}{\partial x_2^*}\Bigg(\rho^*\bigg(\frac{\partial u_2^*}{\partial x_2^*} + \frac{\partial u_2^*}{\partial x_2^*} - \frac{2}{3}(\frac{\partial u_1^*}{\partial x_1^*} + \frac{\partial u_2^*}{\partial x_2^*})\bigg)\Bigg),\\
          \label{eq:34}
          & \frac{\partial}{\partial t^*}(\rho^* e^*) + \frac{\partial}{\partial x_1^*}(\rho^* u_1^*e^*) + \frac{\partial}{\partial x_2^*}(\rho^* u_2^*e^*) = \frac{\gamma}{\mathrm{Pr}\mathrm{Re}}\frac{\partial}{\partial x_1^*}\bigg(\rho^*\frac{\partial T^*}{\partial x_1^*}\bigg) + \frac{\gamma}{\mathrm{Pr}\mathrm{Re}}\frac{\partial}{\partial x_2^*}\bigg(\rho^*\frac{\partial T^*}{\partial x_2^*}\bigg) - \gamma(\gamma - 1)\mathrm{Ma}^2p^*\bigg(\frac{\partial u_1^*}{\partial x_1^*} + \frac{\partial u_2^*}{\partial x_2^*}\bigg)\notag\\
          & + \frac{2\mathrm{Ma}^2\gamma(\gamma - 1)}{\mathrm{Re}}\rho^*\Bigg(\bigg(\frac{\partial u_1^*}{\partial x_1^*}\bigg)^2 + \bigg(\frac{\partial u_2^*}{\partial x_2^*}\bigg)^2 + \frac{1}{2}\bigg(\frac{\partial u_1^*}{\partial x_2^*} + \frac{\partial u_2^*}{\partial x_1^*}\bigg)^2 - \frac{1}{3}\bigg(\frac{\partial u_1^*}{\partial x_1^*} + \frac{\partial u_2^*}{\partial x_2^*}\bigg)^2\Bigg),\\
        \end{align}
        \end{subequations}


        \section{Choose appropriate finite difference operators for the spatial operators}

        \textcolor{blue}{Since the problem is a low-Reynolds number flow, the dispersion error from spatial discretization schemes will rapidly fade away due to high dissipation. Therefore, the simplest approach is to adopt central schemes.}

        For the first-order spatial derivative, we choose the \textcolor{blue}{second-order central scheme} such that for interior grid points $(i_1, i_2), i_1\neq0, i_1\neq N_x, i_2\neq0, i_2\neq N_y$, we have
        \begin{subequations}
          \label{eq:35}
          \begin{align}
            \label{eq:36}
            \frac{\partial (\cdot)^j_{i_1, i_2}}{\partial x_1}\approx\frac{(\cdot)^j_{i_1+1, i_2} - (\cdot)^j_{i_1-1, i_2}}{2\Delta x_1},\\
            \label{eq:37}
            \frac{\partial (\cdot)^j_{i_1, i_2}}{\partial x_2}\approx\frac{(\cdot)^j_{i_1, i_2+1} - (\cdot)^j_{i_1, i_2-1}}{2\Delta x_2},
          \end{align}
        \end{subequations}
        where the coordinates of the grid point are $(x_1, x_2)_{i_1, i_2} = (i_1\Delta x_1, i_2\Delta x_2), i_1 = 0, 1, ..., N_{x_1}, i_2 = 0, 1, ..., N_{x_2}$. $\Delta x_1 = 1/N_{x_1}$ and $\Delta x_2 = 1/N_{x_2}$ are dimensionless spacing on two directions.
        It is worth mentioning that we formulate the central scheme for the second-order spatial derivative in the form of $\partial(\rho \partial u/\partial x)/\partial x$ as follows: First, we evaluate $\rho\partial u/\partial x$ at $(i_1 + 1/2, i_2)$ and $(i_1 - 1/2, i_2)$ via interpolation and central differencing, such that
        \begin{subequations}
          \label{eq:38}
          \begin{align}
            \label{eq:39}
            \rho_{i_1 + \frac{1}{2}, i_2}\bigg(\frac{\partial u}{\partial x_1}\bigg)_{i_1 + \frac{1}{2}, i_2} &\approx \frac{\rho_{i_1, i_2} + \rho_{i_1+1, i_2}}{2}\frac{u_{i_1+1, i_2} - u_{i_1, i_2}}{\Delta x_1},\\
            \label{eq:40}
            \rho_{i_1 - \frac{1}{2}, i_2}\bigg(\frac{\partial u}{\partial x_1}\bigg)_{i_1 - \frac{1}{2}, i_2} &\approx \frac{\rho_{i_1-1, i_2} + \rho_{i_1, i_2}}{2}\frac{u_{i_1, i_2} - u_{i_1-1, i_2}}{\Delta x_1}.
            \end{align}
          \end{subequations}
          Afterwards, we perform central differencing again to obtain the approximation
          \begin{align}
            \label{eq:41}
            \Bigg(\frac{\partial}{\partial x_1}\bigg(\rho\frac{\partial u}{\partial x_1}\bigg)\Bigg)_{i_1, i_2} &\approx \frac{\rho_{i_1 + \frac{1}{2}, i_2}(\partial u/\partial x_1)_{i_1 + \frac{1}{2}, i_2} - \rho_{i_1 - \frac{1}{2}, i_2}(\partial u/\partial x_1)_{i_1 - \frac{1}{2}, i_2}}{\Delta x_1}\notag\\
            &\approx \frac{\rho_{i_1, i_2} + \rho_{i_1+1, i_2}}{2}\frac{u_{i_1+1, i_2} - u_{i_1, i_2}}{(\Delta x_1)^2} - \frac{\rho_{i_1-1, i_2} + \rho_{i_1, i_2}}{2}\frac{u_{i_1, i_2} - u_{i_1-1, i_2}}{(\Delta x_1)^2}.
          \end{align}

          For the points on the domain boundaries, we only care about the values of density since the values of velocities and temperature are specified on the walls.
          To evaluate the normal derivatives in the density boundary conditions \eqref{eq:22}-\eqref{eq:25}, we adopt a one-sided scheme to preserve global second-order accuracy while utilizing only points within the domain.
          The specific schemes are as follows:
          \begin{subequations}
            \label{eq:42}
            \begin{align}
              \label{eq:43}
              \bigg(\frac{\partial u_1}{\partial x_1}\bigg)_{0, i_2} & \approx \frac{-3(u_1)_{0, i_2} + 4(u_1)_{1, i_2} - (u_1)_{2, i_2}}{2\Delta x_1},\\
              \label{eq:44}
              \bigg(\frac{\partial u_2}{\partial x_2}\bigg)_{i_1, 0} & \approx \frac{-3(u_2)_{i_1, 0} + 4(u_2)_{i_1, 1} - (u_2)_{i_1, 2}}{2\Delta x_2},\\
              \label{eq:45}
              \bigg(\frac{\partial u_1}{\partial x_1}\bigg)_{N_{x_1}, i_2} & \approx \frac{3(u_1)_{N_{x_1}, i_2} - 4(u_1)_{N_{x_1}-1, i_2} + (u_1)_{N_{x_1}-2, i_2}}{2\Delta x_1},\\
              \label{eq:46}
              \bigg(\frac{\partial u_2}{\partial x_2}\bigg)_{i_1, N_{x_2}} & \approx \frac{3(u_2)_{i_1, N_{x_2}} - 4(u_2)_{i_1, N_{x_2}-1} + (u_2)_{i_1, N_{x_2}-2}}{2\Delta x_2}.
              \end{align}
            \end{subequations}
            Additionally, to evaluate the tangential derivative of density along the top boundary in \eqref{eq:25}, we use the second-order central schemes on points with $i_1 = 2, 3, ..., N_x-2$ and the second-order one-sided scheme on points with $i_1 = 1, N_x-1$ next to the ill-defined corner points to avoid evaluating the density at the top-left and the top-right corners. 

            \section{Choose appropriate temporal schemes and solution methods}

            In this work, we will choose the 4th-order Runge-Kutta method, since it is easy to implement in our code, has a high temporal accuracy of $O(\Delta t^4)$, and allows us to take a relatively larger timestep compared to forward-Euler method or other low-order explicit schemes.

            \section{Expected approximate CFL limits}

            Since we are working on compressible flow problems, an upper limit of the CFL number for this 2D problem can be approximated by
            \begin{equation}
              \label{eq:47}
              C = \frac{(|u_1|_{\max} + a_{\max})\Delta t}{\Delta x_1} + \frac{(|u_2|_{\max} + a_{\max})\Delta t}{\Delta x_2}.
            \end{equation}
            As we know, for a 1D linear advection PDE $\partial_t u + a\partial_x u = 0$ with the second-order central scheme for spatial derivatives and the 4th-order Runge-Kutta timestepper, the CFL number $a\Delta t/\Delta x$ has a limit of $2\sqrt{2}$.
            So, we can utilize this conclusion to give the same form of upper bound for our CFL number:
            \begin{align}
              \label{eq:48}
              C \leq 2\sqrt{2} \Longrightarrow & \frac{(|u_1|_{\max} + a_{\max})\Delta t}{\Delta x_1} + \frac{(|u_2|_{\max} + a_{\max})\Delta t}{\Delta x_2}\leq 2\sqrt{2}.\notag\\
              \Longrightarrow & \frac{(|u_1^*|_{\max} + \mathrm{Ma}^{-1}\sqrt{T^*_{\max}})\Delta t^*}{\Delta x_1^*} + \frac{(|u_2^*|_{\max} + \mathrm{Ma}^{-1}\sqrt{T^*_{\max}})\Delta t^*}{\Delta x_2^*}\leq 2\sqrt{2}
            \end{align}
            In numerical implementations, we often want to restrict this upper bound by multiplying it with a relaxation parameter, such as $0.6$.

            \section{Pseudo-code of the algorithm for lid-driven flow solver}
            See Algorithm \ref{alg:pseudo-code-solver} on the next page.
            
            \begin{algorithm}
              \caption{RK4-CS Solver for 2D Compressible Lid-Driven Cavity Flow}
              \label{alg:pseudo-code-solver}
\begin{algorithmic}[1]
  \State \textbf{Input:} dimensionless constants
  $\mathrm{Re}$, $\mathrm{Ma}$, $\mathrm{Pr}$, $\gamma$; number of grid points: $N_{x_1}, N_{x_2}$ with grid spacing $\Delta x_1^* = 1/N_{x_1}$, $\Delta x_2^* = 1/N_{x_2}$; timestep $\Delta t^*$ within the stability limit \eqref{eq:48}; number of timesteps $N_t$.
  \State \textbf{Initialize the grid-based variables:} Set $\rho^*(i_1\Delta x_1^*, i_2\Delta x_2^*, 0) = 1$, $u_1^*(i_1\Delta x_1^*, i_2\Delta x_2^*, 0) = 0$, $u_2^*(i_1\Delta x_1^*, i_2\Delta x_2^*, 0) = 0$, and $T^*(i_1\Delta x_1^*, i_2\Delta x_2^*, 0) = 1$ for $i_1 = 0, 1, ..., N_{x_1}$, $i_2 = 0, 1, ..., N_{x_2}$ based on \eqref{eq:26}-\eqref{eq:29}.
  \State \textbf{Define the conservative variables:} Let $\mathbf{U} = (\rho^*, \rho^* u_1^*, \rho^* u_2^*, \rho^* e^*)$ be the variable to be advanced in time, where $e^* = T^*$. 
  \State \textbf{Main timestepping loop:}
  \For{$j = 0$ to $N_t$} (\textbf{the 4th-order Runge-Kutta scheme})
    \State $t^* \leftarrow j\Delta t^*$
    \State Evaluate the CFL number $C \gets \frac{(|u_1^*|_{\max} + \mathrm{Ma}^{-1}\sqrt{T^*_{\max}})\Delta t^*}{\Delta x_1^*} + \frac{(|u_2^*|_{\max} + \mathrm{Ma}^{-1}\sqrt{T^*_{\max}})\Delta t^*}{\Delta x_2^*}$.
        \If{$C > 2\sqrt{2}$}
            \State \textbf{Warning:} Timestep $\Delta t^*$ violates the CFL stability limit.
            \State \textbf{Exit}.
        \EndIf
    \State $\mathbf{U}_0 \gets \mathbf{U}(t^*)$
    \State $\mathbf{k}_1 \gets \Delta t^* \cdot \text{RHS}(\mathbf{U}_0, t^*)$
    \State $\mathbf{k}_2 \gets \Delta t^* \cdot \text{RHS}(\mathbf{U}_0 + 0.5\mathbf{k}_1, t^* + 0.5\Delta t^*)$
    \State $\mathbf{k}_3 \gets \Delta t^* \cdot \text{RHS}(\mathbf{U}_0 + 0.5\mathbf{k}_2, t^* + 0.5\Delta t^*)$
    \State $\mathbf{k}_4 \gets \Delta t^* \cdot \text{RHS}(\mathbf{U}_0 + \mathbf{k}_3, t^* + \Delta t^*)$
    \State $\mathbf{U}(t^* + \Delta t^*) \gets \mathbf{U}_0 + \frac{1}{6}(\mathbf{k}_1 + 2\mathbf{k}_2 + 2\mathbf{k}_3 + \mathbf{k}_4)$
    \EndFor

\hrulefill
\Function{RHS}{$\mathbf{U}, t^*$}
    \State \textbf{Initialization:} Set $d\mathbf{U}/dt$ to be zero.
    \State Extract variables $\rho^*, u_1^*, u_2^*, T^*, e^*, p^*$ from the input $\mathbf{U}$.
    \State Apply the no-slip no-penetration boundary conditions for $u_1^*, u_2^*$ on all walls according to \eqref{eq:19}\eqref{eq:20} using $t^*$.
    \State Apply the isothermal boundary conditions for $T^*$ on all walls according to \eqref{eq:21}.
    

    \For{$i_1 = 1$ to $N_{x_1}-1$}
        \For{$i_2 = 1$ to $N_{x_2}-1$}
            \State Calculate all first and second spatial derivatives on interior points using central schemes \eqref{eq:35}-\eqref{eq:41} to evaluate the RHS for continuity, momentum, and energy equations \eqref{eq:31}-\eqref{eq:34}.
            \State Store results in the corresponding elements of $d\mathbf{U}/dt$.
        \EndFor
    \EndFor
    
    \For{$i_2 = 1$ to $N_{x_2}-1$}
        \State Compute $(\partial u_1^*/\partial x_1^*)_{0,i_2}$ on the left wall using the second-order one-sided scheme \eqref{eq:43}.
        \State $(\frac{d\rho^*}{dt})_{0,i_2} \gets (-\rho^*\partial u_1^*/\partial x_1^*)_{0,i_2}$.
    \EndFor
    \State Repeat Step 25-28 for the bottom ($i_2=0$, \eqref{eq:23}) and the right ($i_1=N_{x_1}$, \eqref{eq:24}) walls using corresponding one-sided schemes.
    
    \For{$i_1 = 2$ to $N_{x_1}-2$}
        \State Compute $(\partial u_2^*/\partial x_2^*)_{i_1,N_{x_2}}$ on the top wall using the second-order one-sided scheme \eqref{eq:46}.
        \State Compute $(\partial \rho^*/\partial x_1^*)_{i_1,N_{x_2}}$ using the second-order central scheme.
        \State $(\frac{d\rho^*}{dt})_{i_1,N_{x_2}} \gets (-\rho^*\partial u_2^*/\partial x_2^*)_{i_1, N_{x_2}} - (u_1^*\partial \rho^*/\partial x_1^*)_{i_1,N_{x_2}}$.
        \EndFor

        \State Compute $(\partial u_2^*/\partial x_2^*)_{1,N_{x_2}}$ on the top wall using the second-order one-sided scheme \eqref{eq:46}.
        \State Compute $(\partial \rho^*/\partial x_1^*)_{1,N_{x_2}}$ using the one-sided second-order scheme.
        \State $(\frac{d\rho^*}{dt})_{1,N_{x_2}} \gets (-\rho^*\partial u_2^*/\partial x_2^*)_{1, N_{x_2}} - (u_1^*\partial \rho^*/\partial x_1^*)_{1,N_{x_2}}$.

        \State Compute $(\partial u_2^*/\partial x_2^*)_{N_{x_1},N_{x_2}}$ on the top wall using the second-order one-sided scheme \eqref{eq:46}.
        \State Compute $(\partial \rho^*/\partial x_1^*)_{N_{x_1},N_{x_2}}$ using the one-sided second-order scheme.
        \State $(\frac{d\rho^*}{dt})_{N_{x_1},N_{x_2}} \gets (-\rho^*\partial u_2^*/\partial x_2^*)_{N_{x_1}, N_{x_2}} - (u_1^*\partial \rho^*/\partial x_1^*)_{N_{x_1},N_{x_2}}$.
        

    \State \Return $\frac{d\mathbf{U}}{dt}$.
\EndFunction
\end{algorithmic}
\end{algorithm}
    
\iffalse{
\begin{figure}[tbp]
  \centering
  \begin{tikzonimage}[width=0.4\linewidth]{solution_implicit_nu_1_dt_1_nx_120_converged.png}%[tsx/show help lines]
    \node at (0.55, -0.05) {(a)};
  \end{tikzonimage} 
  \begin{tikzonimage}[width=0.4\linewidth]{solution_implicit_nu_0.01_dt_1_nx_120_converged.png}%[tsx/show help lines]
    \node at (0.55, -0.05) {(b)};
  \end{tikzonimage}
  \caption{Solutions of Burgers' equation at different $\nu$ and $\Delta t$ using the implicit approach for stability analysis.}
  \label{fig:solution_implicit}
\end{figure} }\fi

\bibliographystyle{plain}
\bibliography{references}
\end{document}
